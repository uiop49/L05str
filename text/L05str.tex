%!TEX program = xelatex
\documentclass{article}
\usepackage[a5paper,hmargin=17mm,tmargin=15mm,bmargin=25mm]{geometry}

\usepackage{ifxetex}
\ifxetex
 \usepackage{fontspec}
 \setmainfont[Scale=1.1]{Arno Pro}
 \setmonofont[Scale=.92]{Consolas}
 \usepackage{unicode-math}              %% пакет для загрузки шрифтов математического режима 
 \setmathfont{[latinmodern-math.otf]}
 \setmathfont[range=\mathit/{latin,Latin}]{Arno Pro Italic}
 \setmathfont[range=up]{Arno Pro}
 \setmathfont[range=\mathup/{latin,Latin}]{Arno Pro}
\else
 \usepackage[utf8]{inputenc}
\fi
\usepackage[russian]{babel}
\usepackage{enumitem}


\begin{document}
\abovedisplayskip=5pt
\belowdisplayskip=3pt
\sloppy

\section*{{\normalsize Лабораторная работа 5} \\Строки}

Цель этой лабораторной работы~--- изучить понятие строки и научиться работать с ними при помощи циклов в JavaScript. 

\bigskip
\noindent\textbf{ВНИМАНИЕ! Файлы называть \texttt{L05-01.js} и т.\,д. Методы строковых объектов не использовать. Не забудьте экспортировать вашу функцию:}\newline
\centerline{\texttt{module.exports = numDots;} \textbf{и т. п.}}

\begin{enumerate}
\item
Напишите функцию \texttt{numDots(s)}, которая возвращает число точек в~строке \texttt{s}.
Например, \texttt{numDots('.x.')} дает \texttt{2}.
\item
Напишите функцию \texttt{ph2f(s)} которая принимает строку \texttt{s} и возвращает результат замены всех вхождений подстроки \texttt{ph} в нее на \texttt{f}. Например, \texttt{ph2f('photographer')} дает \texttt{'fotografer'}.
\item
Напишите функцию \texttt{longestWordLen(s)}, которая дает длину самого длинного слова в строке. Словом считать любую последовательность латинских букв. Например, 
$$
\texttt{longestWord('Quick brown fox jumps, and is gone.')}
$$ 
дает \texttt{5}.
\item
Напишите функцию \texttt{isPalindrome(s)}, которая принимает строку \texttt{s}, состоящую из одних только латинских заглавных букв и пробелов, и возвращает \texttt{true}, если она является палиндромом (читается одинаково слева направо без учета пробелов), и \texttt{false} в противном случае. Например, \texttt{isPalindrome('MADAM I M ADAM')} дает \texttt{true}.  Число действий должно быть пропорционально длине \texttt{s}.
\item
Напишите функцию \texttt{isDelResult(s1, s2)}, которая принимает две строки, и дает \texttt{true}, если \texttt{s2} можно получить из \texttt{s1} удалением некоторых символов, и \texttt{false} в противном случае. Например, 
$$
\texttt{isDelResult('ALABAMA',}{~}\texttt{'ALMA')}
$$
дает \texttt{true}. Число действий должно быть пропорционально суммарной длине двух строк.
\end{enumerate}

\end{document}
